% Compiler ce document 

% package de base
\documentclass[10pt,a4paper]{article}
\usepackage[utf8]{inputenc}
\usepackage{listings}

% langues
\usepackage[usenames,dvipsnames]{xcolor}
\usepackage[francais]{babel}
\usepackage[T1]{fontenc}
\usepackage{amsmath}
\usepackage{amsfonts}
\usepackage{amssymb}
\usepackage{graphicx}
\usepackage{tabularx}
\usepackage{colortbl}
\usepackage[hidelinks]{hyperref} % liens
\usepackage{fancyhdr} % En tetes / bas de page
\usepackage{uarial} % police arial
\usepackage[hidelinks]{hyperref}
\usepackage{xcolor} % Style pour affichage du C
\usepackage{courier} % police pour les listings

% Page de Garde -- Necessite d'installer le package titling, si probleme
% commenter la ligne suivante ainsi que les infos necessaires a la page
% de garde
\usepackage{pageGarde/HEIG_STY}

% commande pour faire des sections sans nombre 
% tout en la rajoutant dans la table des matières
\newcommand\sectionWithoutNumber[1]{\section*{#1} \addcontentsline{toc}{section}{\protect\numberline{}#1}}
\newcommand\subsectionWithoutNumber[1]{\subsection*{#1} \addcontentsline{toc}{subsection}{\protect\numberline{}#1}}
\newcommand\subsubsectionWithoutNumber[1]{\subsection*{#1} \addcontentsline{toc}{subsubsection}{\protect\numberline{}#1}}
% définition de nouvelles couleurs
\definecolor{lightblue}{rgb}{0.8,0.8,0.9}
\definecolor{grossblue}{rgb}{0,0,0.7}
%marge des pages
\setlength{\textwidth}{16cm}
\setlength{\textheight}{24cm}
\setlength{\oddsidemargin}{0cm}
\setlength{\voffset}{-1.5cm}
\setlength{\headheight}{15pt}

% set la police en arial
%% Sans-serif Arial-like fonts
\renewcommand{\rmdefault}{phv} 
\renewcommand{\sfdefault}{phv} 
\usepackage{tabularx}
\usepackage{graphicx}
\usepackage{eurosym}
\usepackage{xspace}
\newcommand{\projectname}[0]{LTANR\xspace} 

% configuration pour des listings
\lstset{ 
  showspaces=false,      
  showstringspaces=false, 
  showtabs=false,               
  tabsize=3,                     
  numbers=left
}

%enlève indentation en début de paragraphe
\setlength\parindent{0pt}

%style de l'en-tête de page
\pagestyle{fancy}

% style pour code en c
\lstdefinestyle{customc}{
  belowcaptionskip=1\baselineskip,
  breaklines=true,
  frame=L,
  xleftmargin=\parindent,
  language=C,
  showstringspaces=false,
  basicstyle=\scriptsize\ttfamily,
  keywordstyle=\bfseries\color{green!40!black},
  commentstyle=\itshape\color{purple!40!black},
  identifierstyle=\color{blue},
  stringstyle=\color{orange},
}

\lstset{escapechar=@,style=customc}
\lstset{inputencoding=utf8/latin1} %affiche les accents dans le listing

% Mise en forme de la page de titre
\author{Domingues Pedrosa João Miguel \\ Haas Loïc}
\title{Protocole applicatif}

% Informations necessaires a la page de garde
% Commenter si probleme de compilation
\acro{SYM}
\cours{Système Mobile}
\date{\today}

%en-tête
\lhead{Domingues João \\ Haas Loïc}
\chead{Protocole Applicatif}
\rhead{\theAcro}

%pied-de-page
\lfoot{HEIG-VD}
\cfoot{\today}
\rfoot{\thepage}

\begin{document}
\maketitle
\newpage
\tableofcontents
\newpage

%Ici commence réelement l'écriture du rapport

\section{Authentification}
Il est tout à fait possible de transmettre même avec une authentification requise.
Pour cela, lors de l'envoie, il fraudais d'abord envoie le nécessaire pour l'authentification, cela implique qu'il faut autoriser la sauvegarde des identifiant à quelque part afin de permettre l'envoie asynchrone.

\section{Threads concurrents}
Il y des problème de synchronisation lors de la modification de l'interface graphique. 
Si par exemple on fait un rotation, on régénéré une activité et la tâche en cours se termine brusquement.

\section{Écriture différée}

\section{Transmission d'objets}
L'inconvénient du infrastructure n'offrant pas de validation de la syntaxe est que la forme de l'information doit être connue par les utilisateurs. 
En effet, on n'arrive pas à identifier la donnée par exemple avec des balises ou autre.
L'avantage est que se type format est léger, compréhensible, il n'a pas besoin de bien connaître la syntaxe pour l'utiliser.
On peut envoyer plusieurs type de données comme des tableaux, images, etc.

\section{Transmission comprimée}
Le taux de compression dépend des données compresser. 
Si on compresser un fichier avec beaucoup de répétition, par exemple du texte, on aura un bon taux.
On peut en moyenne espérer un gain de 10 fois plus petit que le fichier original.

\end{document}
